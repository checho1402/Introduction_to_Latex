\documentclass{article}
\usepackage{amsmath}
\usepackage[T1]{fontenc}
\usepackage[latin1]{inputenc}
\usepackage{subfigure}
\begin{document}
$0^0$ es una expresion indefinida.
Si $a>0$ entonces $a^0=1$ pero $0^a=0.$
Sin embargo, convenir en que $0^0=1$ es adecuado para que
algunas formulas se puedan expresar de manera sencilla,
sin recurrir a casos especiales, por ejemplo
$$\sum _{ini=1}^{fin =100 }{x}$$
$$(x+a)^n=\sum_{k=0}^n \binom{n}{k}x^k a^{n-k}$$
$$\frac{\frac{a}{b}}{\frac{c}{d}}$$
$$\sum_i^n  \sum_i^n   2^{i+j} - 2^{i-j}$$
\end{document}