\documentclass{article}

\usepackage[utf8]{inputenc}
\usepackage[spanish]{babel}
\usepackage[total={18cm,21cm},top=2cm, left=2cm]{geometry}
\usepackage{hyperref}
\usepackage{lipsum}
\usepackage{rotating}
\usepackage{graphicx}
\usepackage{amsmath,amssymb,amsfonts,latexsym,cancel}
\usepackage[x11names,table]{xcolor}
\usepackage{subfigure}

\renewcommand{\listtablename}{Índice de tablas}
\AtBeginDocument{\renewcommand\tablename{Tabla}}

\title{Ejercicio 2 Tablas y Figuras}
\author{Sergio Leandro Ramos Villena}

\begin{document}

\maketitle


\section{Configuración}
	\begin{itemize}
		\item Título, autor y fecha
		\item Establecer el tamaño de la página a 18x21cm con márgen a los costados de 2cm.
		\item Renombrar \textbf{Cuadros} por \textbf{Tablas} colocando el siguiente comando en el preámbulo (la parte antes del begin document).
			\begin{verbatim}

			\renewcommand{\listtablename}{´Indice de tablas}
			\AtBeginDocument{\renewcommand\tablename{Tabla}}

			\end{verbatim}
		\item Utilizar el paquete \textit{hyperref}
	\end{itemize}
\pagebreak
\section{Tablas}
 \begin{itemize}
 
	\item Tabla \ref{table:columnas}: Coloque las siguientes ecuaciones en una tabla con dos columnas:
		\begin{table}[h!]
		\centering
			\begin{tabular}{|c|c|}\hline
	

			$\displaystyle \frac{1}{1+\frac{1}{1+\frac{1}{1+\frac{1}{1+...}}}}$ & $\displaystyle \sum _{i}^{n-1}\sum _{j=i+1}^{n} f(i,j)e^{-\pi}$ \\\hline

			\end{tabular}	
		\caption{tabla de dos columnas}\label{table:columnas}
		\end{table}
	
	\item Tabla \ref{table:x}: Utilice el método de cellcolor para dibujar una X en una tabla de tama˜no de 7x7.
		\begin{table}[h!]
		\centering
			\begin{tabular}{|c|c|c|c|c|c|c|}\hline
			\cellcolor[gray]{0} &&&&&&\cellcolor[gray]{0}\\\hline
			&\cellcolor[gray]{0}&&&&\cellcolor[gray]{0}&\\\hline
			&&\cellcolor[gray]{0}&&\cellcolor[gray]{0}&&\\\hline
			&&&\cellcolor[gray]{0}&&&\\\hline
			&&\cellcolor[gray]{0}&&\cellcolor[gray]{0}&&\\\hline
			&\cellcolor[gray]{0}&&&&\cellcolor[gray]{0}&\\\hline
			\cellcolor[gray]{0}&&&&&&\cellcolor[gray]{0}\\\hline
			\end{tabular}	
		\caption{X en una tabla de tama˜no de 7x7.}\label{table:x}
		\end{table}
	
	\item Coloque la siguiente Tabla \ref{table:semana}:
		\begin{table}[h!]
		\centering
			\begin{tabular}{|c|c|c|c|c|c|c|c|}\hline
				&\multicolumn{6}{c}{Dias de la semana}&\\\hline
				&Lunes&Martes&Miércoles&Jueves&Viernes&Sábado&Domingo\\\hline
				\begin{sideways}Presente\end{sideways}&Si&No&Si&No&Si&No&No \\\hline
			\end{tabular}	
		\caption{Asistencias del día de la semana.}\label{table:semana}
		\end{table}
	\item Coloque las referencias correctas en donde aparece el símbolo de interrogación.Y coloque a las tablas un caption
en la parte inferior.

 \end{itemize}
\pagebreak
\section{Figuras}
	\begin{enumerate}
		\item En la Figura \ref{figure:centro}, logotipo de la UCSP con un ancho de 5cm. y centralizado.
			\begin{figure}[h!]
				\begin{center}
				\caption{Logotipo en el centro}\label{figure:centro}
					\includegraphics[width =5cm]{D:/Descargas/Logotipo.png}

				\end{center}
			\end{figure}
	\item En la Figura \ref{figure:derecha},, logotipo de la UCSP con un ancho de 5cm y una altura de 8cm y alineado a la derecha de la
página.	

		\begin{figure}[h!]
			\parbox{13cm}{.}\parbox{5cm}{ \caption{logotipo a la derecha}\label{figure:derecha}
				\includegraphics[width =5cm,height =8cm]{D:/Descargas/Logotipo.png}}


		\end{figure}
	\item En la Figura  \ref{table:doscelulares}, colocará dos gráficos juntos (a) y (b). El (a) será la imágen de un Motorola G6 de 4cm de ancho.
El (b) sería la imágen de un samsung galaxy S10 con los mismos 4cm de ancho en la imagen.

\textbf{Abajo}

		\begin{figure}[h!]
		\centering
			\subfigure[Motorola G6]{\includegraphics[width =4cm]{D:/Descargas/motog6.png}}
			\subfigure[Samsung galaxy S10]{\includegraphics[width =4cm]{D:/Descargas/Samsung S10.png}}


		\caption{Celulares Motorola y Samsung.}\label{table:doscelulares}

		\end{figure}
	

	\end{enumerate}

\end{document}
