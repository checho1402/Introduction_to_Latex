\documentclass{article}
\usepackage[top=5mm,left=5mm,right=5mm,bottom=5mm]{geometry}
\usepackage[utf8]{inputenc}
\usepackage[spanish]{babel}

\usepackage{xurl}
\usepackage{showframe}
\usepackage{pstricks}
\usepackage{multicol}

\title{Metodología del estudio 2020-I \\ Ejercicio 1 de Latex}

\author{Sergio Leandro Ramos Villena}
\date{Universidad Católica San Pablo \\ \url{sergio.ramos.villena@ucsp.edu.pe} }


\begin{document}

\maketitle
\tableofcontents

\section{Transcribir un texto}
Transcribir la primera parte del texto del artículo (hasta antes de la sección
\textit{LAS INVESTIGACIONES}).

\url{https://ucsp.edu.pe/cuatro-titulados-mediante-paper-o-articulo-d
e-investigacion-en-la-ucsp/}

No colocar imágenes

	{\bf Son egresados de la carrera de Ingeniería Electrónica y de Telecomunicaciones}

{\bf Esther Rosas Bermejo, Alana Núñez Flores, Alexander Paredes Choque y Giancarlo Murillo Arenas son los primeros titulados de la Universidad Católica San Pablo bajo la modalidad de paper o artículo de investigación}. Los egresados de la {\bf Escuela de Ingeniería Electrónica y de Telecomunicaciones (IET) de la UCSP presentaron estas investigaciones en congresos internacionales, por ello fueron escritas y expuestas en inglés}.

Esta es una de las {\bf dos nuevas modalidades de titulación} que ha implementado la Escuela de Ingeniería Electrónica y de Telecomunicaciones de la UCSP, este año. {\bf La otra es por emprendimiento tecnológico}. Ambas se suman a la de tesis y por suficiencia profesional. {\bf La Escuela de IET ha podido incluir estas nuevas opciones ya que obtuvo su acreditación por el SINEACE en agosto de 2016. Es la única carrera de ingeniería de Arequipa a la que el organismo estatal le ha reconocido un alto estándar de calidad educativa entre ese año a setiembre de 2019}.

Son distintas {\bf las ventajas que el paper ofrece a los futuros profesionales}. Entre las principales está el reconocimiento internacional, el ahorro de tiempo y el mayor peso académico, porque para que un artículo científico sea difundido en un congreso o publicado de manera indexada, es revisado por comités científicos ajenos a la universidad de donde proviene, precisó el {\bf Dr. Gonzalo Fernández Del Carpio, responsable de Grados y Títulos de la Escuela de IET-UCSP}.

{\bf Para los nuevos ingenieros esta nueva opción de grado les resultó más que beneficiosa} porque es más sencilla en cuanto a procedimientos, les ha permitido iniciarse en la investigación estando ellos en pregrado y que su trabajo sea conocido dentro y fuera de Perú e incluso tomado como referencia por otros investigadores. Académicamente, coinciden que les será de utilidad para futuros estudios de posgrado.

“Si bien el {\bf paper es más sencillo en cuanto a formato porque tienen un número limitado de páginas, es incluso más riguroso que la tesis de pregrado}, porque su fin es alcanzar que la publicación sea indexada internacionalmente, para lo que es necesario que signifique un aporte novedoso sobre el tema que se está tratando, es redactado y presentado en inglés, incluso, sus autores deben revisar mucha bibliografía y tener gran capacidad de síntesis”, comentó el Dr. Fernández Del Carpio.

Los jóvenes ingenieros trabajaron en {\bf un sistema de comunicación acoplado al cuerpo}, en {\bf antenas tejidas}, analizaron la {\bf propagación del sonido para redes de altavoces de emergencia} y en un {\bf sensor sumergible para evaluar la calidad del pisco}. Estas investigaciones les tomaron entre uno a un año y medio. Las realizaron como parte de sus cursos de metodología de investigación y de proyecto de tesis, y en los proyectos de investigación del {\bf Departamento de Ingeniería Eléctrica y Electrónica de la UCSP} con la asesoría de sus profesores.

Los alumnos presentaron sus papers en los congresos: {\bf XXV International Conference on Electronics, Electrical Engineering and Computing 2018, ANDESCON 2016} y {\bf Latin America Microwave Conference 2018. Estos eventos son avalados por el IEEE, que es la organización de profesionales en tecnología más importante a nivel mundial}. Los artículos de los egresados de la UCSP se encuentran en el repositorio digital del IEEE.

\section{Alterar configuraciones}
\section{Colocar configuraciones}
\begin{enumerate}
	\item \fcolorbox{black}{orange}{	\color{white} Latex}
	\item {\bf 1 Punto}
	\item {\sc Tinty}
\end{enumerate}
\begin{multicols}{2}
{\it Este} es mi {\tiny primer ejemplo} en {\Huge LaTeX} en donde voy a aprender
bastantes cosas. Por ejemplo, a escribir con {\blue diferentes} {\red colores} y diferentes {\sc formatos}. El \underline{siguiente dibujo}
se hizo utilizando {\tt verbatim}.
\begin{verbatim}
\ O /
 | |
 / \
// \\
\end{verbatim}
\end{multicols}
\end{document}